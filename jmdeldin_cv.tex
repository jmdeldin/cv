\documentclass[10pt]{xecv}
% \overfullrule=5pt

\author{Jon-Michael Anthony Deldin}
% \renewcommand{\streetfield}{}
% \renewcommand{\cityfield}{}
% \renewcommand{\phone}{}

\renewcommand{\streetfield}{}
\renewcommand{\cityfield}{}
\renewcommand{\phone}{}
\renewcommand{\email}{}

\usepackage{paracol}
\setlength{\columnsep}{3in}
\makeatletter
\hypersetup{pdfauthor={\@author}, pdftitle={\@author's Curriculum Vitae}}

\begin{document}

\begin{paracol}{2}
  \advance\columnwidth 2.5in \hsize\columnwidth
  \xesection{Experience}
  % @{} suppresses spaces
  \begin{tabularx}{\columnwidth}{@{}@{}l X@{}@{}}
    \daterange{2010-02}{present} &
    \job{Software Developer}{Biomimicry 3.8}{Missoula, MT} \\
    & \begin{joblist}
      \item Developed the Biomimicry Student Design Challenge website. It
        was a huge success with over 500 students registering in the
        first year and over 1,200 students registering in the second year.

      % 1st year: >500 students / 116 teams / 50 submissions
      % 2nd year: 1265 students / 339 teams / 68 submissions

      \begin{itemize}
      \item I was the sole developer responsible for the client-side,
        server-side, and system administration.
      \item I developed the site with test-driven development using Ruby
        on Rails, RSpec, Capybara (integration testing), Postgres,
        Heroku, Ticketmaster/IATS (payment gateway), Memcached, and
        Amazon Cloudfront CDN.
    \end{itemize}

    \item On AskNature.org, a database of nature-inspired designs, I optimized
      SQL queries for drastic speedups, implemented search
      index health monitoring, and implemented automated spam prevention and
      detection tools.

      % Our team won the 2010 Earth Award and was a finalist in the 2011
      % INDEX Awards.

    \item Designed the service-oriented architecture and internal APIs
      for AskNature 2.0.
    \item Developed backup, auditing, analytics, visualization, and
      reporting tools using Ruby, bash, Graphviz, and R.
    \end{joblist}
  \end{tabularx}

  \begin{tabularx}{\columnwidth}{@{}l X@{}}
    \daterange{2005-05}{present} &
    \job{Freelance Web Developer}{jmdeldin.com}{Missoula, MT} \\
    & {\small I designed websites, transcribed mockups into XHTML and CSS,
      developed content management systems, contributed to open-source
      projects, and managed CentOS and Debian servers.}\vfill
  \end{tabularx}

  \begin{tabularx}{\columnwidth}{@{}l X@{}}
    \daterange{2011-08}{2011-12} &
    \job{Program Manager}{Spectral Fusion Designs}{Missoula, MT} \\
    & {\small As part of a semester assignment, I managed a student-run web
      design group. I spearheaded upgrading development practices to current
      standards in addition to gathering requirements, setting timelines, and
      managing client projects. } \vfill
  \end{tabularx}

  \begin{tabularx}{\columnwidth}{@{}l X@{}}
    \daterange{2011-08}{2011-12} &
    \job{Teaching Assistant}{University of Montana}{Missoula, MT} \\
    & {\small I taught students in a bioinformatics lab practical Perl and
      {\sc unix} skills for parsing and analyzing sequences, blast output, and other
      data.} \vfill
  \end{tabularx}

  \begin{tabularx}{\columnwidth}{@{}l X@{}}
    \daterange{2009-06}{2010-11} &
    \job{Web Developer}{Regional Learning Project}{Missoula, MT} \\
    & {\small I illustrated maps, produced interactive websites using PHP and
      JavaScript, and managed older sites with Python, \texttt{sed}, and
      friends.} \\
  \end{tabularx}

  \xesection{Education}
  \begin{tabularx}{\columnwidth}{@{}l X@{}}
    \daterange{2009-08}{2013-05} & \uni{University of Montana-Missoula} \\
    & \edtitle{Master's degree in computer science} \\
    & {\small Emphasis in human-computer interaction and machine learning}\\
    % & {\small \emph{Emphasis:} Automated content evaluation\amp mood
    %   classification} \\
    & \vfill \\
  \end{tabularx}

  \begin{tabularx}{\columnwidth}{@{}l X@{}}
    \daterange{2005-08}{2009-08} & \uni{University of Great Falls} \\
    & \edtitle{B.A. Biology \& B.A. Chemistry} \\
    & {\small \emph{Final project:} Literature review on systemic sclerosis}
  \end{tabularx}

\xesection{Skills}
\begin{tabularx}{\columnwidth}{@{}l X@{}}
  Languages (proficient) & JavaScript, Perl, PHP, Python, Ruby, sh, SQL \\
  Languages (familiar) &  C, C++, Java, Objective-C, OCaml, Scheme \\
  % Markup languages  & CSS, HTML, \LaTeX, XHTML, XML, XSLT, YAML\\
  % Databases         & MySQL, Postgres, SQLite \\
  % Operating systems & Debian, CentOS, Fedora, Mac OS X, Ubuntu \\
  % Web servers       & Apache HTTPD, Lighttpd, nginx, Thin \\
  Techniques & User-centered design, test-driven development \\
\end{tabularx}

  \switchcolumn
  \footnotesize
  % \subsection*{Professional Interests}
  % Machine learning, natural language processing, human-computer
  % interaction, service-oriented architecture, and data visualization

  % Designing highly-usable and useful products, building tools to increase
  % productivity, visualizing data, and designing \\developer-friendly APIs.

  \subsection*{Selected Projects}
  The following projects and others can be viewed at
  \href{http://www.jmdeldin.com/?utm_source=cv&utm_medium=pdf&utm_content=otherproj&utm_campaign=cv-otherproj}{jmdeldin.com}
  and on GitHub
  \href{https://github.com/jmdeldin}{@jmdeldin}.

  \subsubsection*{NLP Spam Classification}
  For AskNature.org, I developed a na\"{i}ve Bayes classifier to detect spam
  accounts with 95.7\% precision. I implemented a parallel
  cross-validation algorithm to compare $n$-gram models for word,
  character, and phonetic grams.

  \subsubsection*{GrizSpace iPhone App}
  In Human-Computer Interaction, my team developed an iPhone app for finding
  classes on our campus. I designed the architecture, normalized the database,
  wrote unit tests, parsed course data with Ruby, and trained my partners in
  Git.

  \subsubsection*{Interactive Hyperelasticity}
  For Computer Simulation\amp Modeling, I created a web app to investigate the
  impact of boundary conditions and other parameters on a hyperelastic
  material. I used FEniCS to solve a 2D model of hyperelasticity, and then I
  plotted the resultant deformation via Matplotlib. This was all deployed to
  an Amazon EC2 instance.

  \subsubsection*{Enron Corpus Visualization}
  For my data visualization course, I dug through the Enron email corpus and
  produced standalone posters of my findings. I used Ruby for preprocessing
  2.5 GB of emails into a Postgres database, Python and Matplotlib for time
  series, and Cytoscape for visualizing networks.

  \subsubsection*{RnaSec}
  I developed a Ruby library for manipulating RNA secondary structures
  as tree data structures. Additionally, I developed a proof-of-concept
  genetic algorithm to compare ligands represented as trees.

  \subsubsection*{Missoula Adoptables}
  In Software Engineering, I designed and developed a mashup website of
  adoptable pets. I converted my Photoshop mockup to CSS and HTML, developed a
  custom Model-View-Controller framework in \\PHP, and wrote the frontend.

  \subsection*{Publications}
  JM Deldin and M. Schuknecht.
  \textit{Biologically Inspired Design: Computational Methods and Tools},
  chapter The AskNature Database: Enabling Solutions in Biomimetic
  Design. Springer, forthcoming 2013.
\end{paracol}

\end{document}
